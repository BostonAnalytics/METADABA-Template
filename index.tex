% Options for packages loaded elsewhere
\PassOptionsToPackage{unicode}{hyperref}
\PassOptionsToPackage{hyphens}{url}
\PassOptionsToPackage{dvipsnames,svgnames,x11names}{xcolor}
%
\documentclass[
  letterpaper,
  DIV=11,
  numbers=noendperiod]{scrartcl}

\usepackage{amsmath,amssymb}
\usepackage{iftex}
\ifPDFTeX
  \usepackage[T1]{fontenc}
  \usepackage[utf8]{inputenc}
  \usepackage{textcomp} % provide euro and other symbols
\else % if luatex or xetex
  \usepackage{unicode-math}
  \defaultfontfeatures{Scale=MatchLowercase}
  \defaultfontfeatures[\rmfamily]{Ligatures=TeX,Scale=1}
\fi
\usepackage{lmodern}
\ifPDFTeX\else  
    % xetex/luatex font selection
\fi
% Use upquote if available, for straight quotes in verbatim environments
\IfFileExists{upquote.sty}{\usepackage{upquote}}{}
\IfFileExists{microtype.sty}{% use microtype if available
  \usepackage[]{microtype}
  \UseMicrotypeSet[protrusion]{basicmath} % disable protrusion for tt fonts
}{}
\makeatletter
\@ifundefined{KOMAClassName}{% if non-KOMA class
  \IfFileExists{parskip.sty}{%
    \usepackage{parskip}
  }{% else
    \setlength{\parindent}{0pt}
    \setlength{\parskip}{6pt plus 2pt minus 1pt}}
}{% if KOMA class
  \KOMAoptions{parskip=half}}
\makeatother
\usepackage{xcolor}
\setlength{\emergencystretch}{3em} % prevent overfull lines
\setcounter{secnumdepth}{-\maxdimen} % remove section numbering
% Make \paragraph and \subparagraph free-standing
\makeatletter
\ifx\paragraph\undefined\else
  \let\oldparagraph\paragraph
  \renewcommand{\paragraph}{
    \@ifstar
      \xxxParagraphStar
      \xxxParagraphNoStar
  }
  \newcommand{\xxxParagraphStar}[1]{\oldparagraph*{#1}\mbox{}}
  \newcommand{\xxxParagraphNoStar}[1]{\oldparagraph{#1}\mbox{}}
\fi
\ifx\subparagraph\undefined\else
  \let\oldsubparagraph\subparagraph
  \renewcommand{\subparagraph}{
    \@ifstar
      \xxxSubParagraphStar
      \xxxSubParagraphNoStar
  }
  \newcommand{\xxxSubParagraphStar}[1]{\oldsubparagraph*{#1}\mbox{}}
  \newcommand{\xxxSubParagraphNoStar}[1]{\oldsubparagraph{#1}\mbox{}}
\fi
\makeatother


\providecommand{\tightlist}{%
  \setlength{\itemsep}{0pt}\setlength{\parskip}{0pt}}\usepackage{longtable,booktabs,array}
\usepackage{calc} % for calculating minipage widths
% Correct order of tables after \paragraph or \subparagraph
\usepackage{etoolbox}
\makeatletter
\patchcmd\longtable{\par}{\if@noskipsec\mbox{}\fi\par}{}{}
\makeatother
% Allow footnotes in longtable head/foot
\IfFileExists{footnotehyper.sty}{\usepackage{footnotehyper}}{\usepackage{footnote}}
\makesavenoteenv{longtable}
\usepackage{graphicx}
\makeatletter
\def\maxwidth{\ifdim\Gin@nat@width>\linewidth\linewidth\else\Gin@nat@width\fi}
\def\maxheight{\ifdim\Gin@nat@height>\textheight\textheight\else\Gin@nat@height\fi}
\makeatother
% Scale images if necessary, so that they will not overflow the page
% margins by default, and it is still possible to overwrite the defaults
% using explicit options in \includegraphics[width, height, ...]{}
\setkeys{Gin}{width=\maxwidth,height=\maxheight,keepaspectratio}
% Set default figure placement to htbp
\makeatletter
\def\fps@figure{htbp}
\makeatother

\KOMAoption{captions}{tableheading}
\makeatletter
\@ifpackageloaded{caption}{}{\usepackage{caption}}
\AtBeginDocument{%
\ifdefined\contentsname
  \renewcommand*\contentsname{Table of contents}
\else
  \newcommand\contentsname{Table of contents}
\fi
\ifdefined\listfigurename
  \renewcommand*\listfigurename{List of Figures}
\else
  \newcommand\listfigurename{List of Figures}
\fi
\ifdefined\listtablename
  \renewcommand*\listtablename{List of Tables}
\else
  \newcommand\listtablename{List of Tables}
\fi
\ifdefined\figurename
  \renewcommand*\figurename{Figure}
\else
  \newcommand\figurename{Figure}
\fi
\ifdefined\tablename
  \renewcommand*\tablename{Table}
\else
  \newcommand\tablename{Table}
\fi
}
\@ifpackageloaded{float}{}{\usepackage{float}}
\floatstyle{ruled}
\@ifundefined{c@chapter}{\newfloat{codelisting}{h}{lop}}{\newfloat{codelisting}{h}{lop}[chapter]}
\floatname{codelisting}{Listing}
\newcommand*\listoflistings{\listof{codelisting}{List of Listings}}
\makeatother
\makeatletter
\makeatother
\makeatletter
\@ifpackageloaded{caption}{}{\usepackage{caption}}
\@ifpackageloaded{subcaption}{}{\usepackage{subcaption}}
\makeatother

\ifLuaTeX
  \usepackage{selnolig}  % disable illegal ligatures
\fi
\usepackage{bookmark}

\IfFileExists{xurl.sty}{\usepackage{xurl}}{} % add URL line breaks if available
\urlstyle{same} % disable monospaced font for URLs
\hypersetup{
  pdftitle={Welcome to Preparatory Labs for Databases},
  colorlinks=true,
  linkcolor={blue},
  filecolor={Maroon},
  citecolor={Blue},
  urlcolor={Blue},
  pdfcreator={LaTeX via pandoc}}


\title{Welcome to Preparatory Labs for Databases}
\author{}
\date{}

\begin{document}
\maketitle


\includegraphics{index_files/mediabag/web-database-1.gif}

\subsection{Course Overview}\label{course-overview}

\subsubsection{Course Description}\label{course-description}

This SQL Bootcamp is designed to teach students, particularly those in
applied business analytics and data science, the foundations and
advanced techniques of SQL using both \textbf{MySQL} and
\textbf{PostgreSQL}. Over nine modules, you will master fundamental SQL
concepts such as querying data, working with joins, stored procedures,
transactions, and more. Each module is accompanied by self-guided
learning tasks, quizzes, and practical assessments to ensure a deep
understanding of database concepts.

\subsubsection{Learning Outcomes}\label{learning-outcomes}

Upon completing the SQL Bootcamp, you will be able to:

\begin{itemize}
\tightlist
\item
  Understand and write basic SQL queries.
\item
  Retrieve, manipulate, and manage data efficiently.
\item
  Work with multiple tables using joins and set operations.
\item
  Create and manage databases, tables, and indexes.
\item
  Understand and implement stored procedures, functions, triggers, and
  transactions.
\end{itemize}

\subsubsection{Total Time Required}\label{total-time-required}

\begin{itemize}
\tightlist
\item
  \textbf{Instructor-led learning}: 2 hours (each module for 20 minutes)
\item
  \textbf{Self-guided learning and assessments}: 4 hours combined
\end{itemize}

\subsection{SQL Course Outline}\label{sql-course-outline}

\subsubsection{Getting Started with SQL}\label{getting-started-with-sql}

\begin{itemize}
\tightlist
\item
  \textbf{Objectives}:

  \begin{itemize}
  \tightlist
  \item
    Introduction to relational databases, setting up MySQL and
    PostgreSQL, understanding SQL syntax and data types.
  \end{itemize}
\item
  \textbf{Lectures}:

  \begin{enumerate}
  \def\labelenumi{\arabic{enumi}.}
  \tightlist
  \item
    \textbf{Introduction to MySQL and PostgreSQL}: Overview of popular
    relational databases and setting up local environments.
  \item
    \textbf{Data Types}: Understanding core SQL data types (e.g.,
    \texttt{int}, \texttt{varchar}, \texttt{date}, etc.).
  \item
    \textbf{Creating Databases and Tables}: Steps to create a new
    database and define tables.
  \item
    \textbf{Inserting, Updating, and Deleting Records}: Basic SQL
    operations for manipulating records.
  \end{enumerate}
\end{itemize}

\subsubsection{Retrieving Data with SQL}\label{retrieving-data-with-sql}

\begin{itemize}
\tightlist
\item
  \textbf{Objectives}:

  \begin{itemize}
  \tightlist
  \item
    Learn to write SQL queries to retrieve data from a database.
  \end{itemize}
\item
  \textbf{Lectures}:

  \begin{enumerate}
  \def\labelenumi{\arabic{enumi}.}
  \tightlist
  \item
    \textbf{SELECT Statements}: Introduction to the \texttt{SELECT}
    statement for querying data.
  \item
    \textbf{WHERE Clauses and Filtering Data}: Using \texttt{WHERE} to
    filter query results based on conditions.
  \item
    \textbf{Sorting Data with ORDER BY, LIMIT, and OFFSET}: Techniques
    for ordering and limiting results.
  \item
    \textbf{Using Variables in Queries}: Working with variables to
    enhance query flexibility.
  \end{enumerate}
\end{itemize}

\subsubsection{Modifying Data with SQL}\label{modifying-data-with-sql}

\begin{itemize}
\tightlist
\item
  \textbf{Objectives}:

  \begin{itemize}
  \tightlist
  \item
    Master techniques for modifying data in SQL databases.
  \end{itemize}
\item
  \textbf{Lectures}:

  \begin{enumerate}
  \def\labelenumi{\arabic{enumi}.}
  \tightlist
  \item
    \textbf{INSERT, UPDATE, DELETE Statements}: Essential commands for
    data manipulation.
  \item
    \textbf{Error Handling and Constraints}: Handling constraints like
    \texttt{PRIMARY\ KEY} and \texttt{FOREIGN\ KEY}, and error
    management.
  \item
    \textbf{Managing NULL Values and Default Values}: Handling missing
    or default values in SQL.
  \item
    \textbf{String Operations in SQL}: Useful string functions like
    \texttt{CONCAT}, \texttt{SUBSTRING}, etc.
  \end{enumerate}
\end{itemize}

\subsubsection{Joins and Advanced
Queries}\label{joins-and-advanced-queries}

\begin{itemize}
\tightlist
\item
  \textbf{Objectives}:

  \begin{itemize}
  \tightlist
  \item
    Learn to retrieve and work with data across multiple tables using
    SQL joins.
  \end{itemize}
\item
  \textbf{Lectures}:

  \begin{enumerate}
  \def\labelenumi{\arabic{enumi}.}
  \tightlist
  \item
    \textbf{INNER JOIN, LEFT JOIN, RIGHT JOIN, FULL JOIN}: Understanding
    different types of joins and their uses.
  \item
    \textbf{UNION and UNION ALL}: Combining results from multiple
    queries.
  \item
    \textbf{Grouping and Aggregating Data with GROUP BY}: Techniques for
    grouping data and using aggregate functions like \texttt{SUM},
    \texttt{COUNT}, etc.
  \item
    \textbf{HAVING Clauses}: Filtering aggregated data with
    \texttt{HAVING}.
  \end{enumerate}
\end{itemize}

\subsubsection{Managing Database
Structures}\label{managing-database-structures}

\begin{itemize}
\tightlist
\item
  \textbf{Objectives}:

  \begin{itemize}
  \tightlist
  \item
    Learn to modify and optimize database structures for improved
    performance.
  \end{itemize}
\item
  \textbf{Lectures}:

  \begin{enumerate}
  \def\labelenumi{\arabic{enumi}.}
  \tightlist
  \item
    \textbf{ALTER TABLE and MODIFY Operations}: Altering table
    structures and modifying columns.
  \item
    \textbf{Creating and Deleting Indexes}: Indexing strategies for
    faster data retrieval.
  \item
    \textbf{Optimizing Queries for Performance}: Techniques to improve
    query execution times.
  \item
    \textbf{Handling Large Datasets and Backups}: Managing large
    datasets, backups, and recovery
  \end{enumerate}
\end{itemize}

\subsubsection{Stored Procedures and
Triggers}\label{stored-procedures-and-triggers}

\begin{itemize}
\tightlist
\item
  \textbf{Objectives}:

  \begin{itemize}
  \tightlist
  \item
    Introduction to automation and server-side programming with stored
    procedures and triggers.e
  \end{itemize}
\item
  \textbf{Lectures}:

  \begin{enumerate}
  \def\labelenumi{\arabic{enumi}.}
  \tightlist
  \item
    \textbf{Creating and Calling Stored Procedures}: Writing reusable
    procedures for SQL operations.
  \item
    \textbf{Creating and Managing Triggers}: Automating actions with
    database triggers.
  \item
    \textbf{Error Handling in Stored Procedures}: Managing errors within
    stored procedures.
  \item
    \textbf{Performance Considerations}: Best practices for efficient
    stored procedures and triggers.
  \end{enumerate}
\end{itemize}




\end{document}
